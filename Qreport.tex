%--------------------------------------------------------------------------------------------------
%  This is tex/pdf report generator to produce Long-Term Target Performance Table
%  to replace the existing one at http://www.wano.org/xwp10webapp/reports/TableReport.pdf?x=774741
%  Please check that QReport.r has been started beforehand
%--------------------------------------------------------------------------------------------------

\documentclass{article}
\usepackage{graphicx}
\usepackage{hyperref}
\usepackage{amsmath}
\usepackage{times}
\usepackage{multirow}

\setlength\parindent{0pt}

\usepackage{geometry}
 \geometry{
 letterpaper,
 total={170mm,257mm},
 left=20mm,
 top=5mm,
 }

%\textwidth=18cm
%\textheight=26cm
%\parskip=.3cm
%\oddsidemargin=.1in
%\evensidemargin=.1in
%\headheight=-.3in

\usepackage{Sweave}
\begin{document}

%------------------------------------------------------------
\title{Regional Centre Target Performance Reports}
%------------------------------------------------------------

%\author{WANO London Office}
%\date{}






%\maketitle

\begin{Schunk}
\begin{Soutput}
Results through 2017Q1 have been generated on Tue Jun 06 15:34:27 2017
\end{Soutput}
\end{Schunk}

%\tableofcontents
%------------------------------------------------------------
\section*{Long-Term Targets Performance Table}
%------------------------------------------------------------
\begin{center}
\begin{tabular}{|p{5cm}||p{5cm}|p{5cm}|}
\hline

% ---------------------- Worldwide_LTT -------------------------
Key Indicator (Reactor types combined for applicable indicators) & Percentage Achieving Individual Target (Objective is 100 percent) & Percentage Achieving Industry Target (Objective is 75 percent (100 percent for SSPI)) \\
\hline\hline FLR & 85.3 & 57.9 \\
\hline 
CRE & 91.3 & 77.8 \\
\hline 
TISA & 85.8 & 70.6 \\
\hline 
US7 & 84.7 & 60.9 \\
\hline 
SSPI & 98.5 & 96.3 \\
\hline \end{tabular}
\end{center}

% ------------------------ LTT by Regional Centre -----------------------
\begin{center} \begin{tabular}{|p{1.5cm}||p{2.5cm}||p{2.5cm}|p{2.5cm}||p{2.5cm}|p{2.5cm}|} \hline\multicolumn{2}{|c||}{\textbf{ Atlanta Centre }} & \multicolumn{2}{|c||}{Individual Target Related Performance} & \multicolumn{2}{|c|}{Industry Target Related Performance} \\
\hline 
Key Indicator & Number of units (systems for SSPI) with qualified results & Number of units (systems for SSPI) achieving individual target & Percentage of regional centre units achieving individual target & Number of units achieving industry target & Percentage of regional centre units achieving industry target \\
\hline 
FLR & 123 & 108 & 87.8 & 74 & 60.2 \\
\hline 
CRE & 123 & 118 & 95.9 & 105 & 85.4 \\
\hline 
TISA & 68 & 68 & 100 & 65 & 95.6 \\
\hline 
US7 & 123 & 108 & 87.8 & 74 & 60.2 \\
\hline 
SSPI & 314 & 306 & 97.5 & 115 & 93.5 \\
\hline 
\end{tabular} \end{center}\begin{center} \begin{tabular}{|p{1.5cm}||p{2.5cm}||p{2.5cm}|p{2.5cm}||p{2.5cm}|p{2.5cm}|} \hline\multicolumn{2}{|c||}{\textbf{ Moscow Centre }} & \multicolumn{2}{|c||}{Individual Target Related Performance} & \multicolumn{2}{|c|}{Industry Target Related Performance} \\
\hline 
Key Indicator & Number of units (systems for SSPI) with qualified results & Number of units (systems for SSPI) achieving individual target & Percentage of regional centre units achieving individual target & Number of units achieving industry target & Percentage of regional centre units achieving industry target \\
\hline 
FLR & 70 & 59 & 84.3 & 49 & 70 \\
\hline 
CRE & 69 & 62 & 89.9 & 52 & 75.4 \\
\hline 
TISA & 25 & 24 & 96 & 20 & 80 \\
\hline 
US7 & 69 & 66 & 95.7 & 56 & 81.2 \\
\hline 
SSPI & 163 & 159 & 97.5 & 65 & 94.2 \\
\hline 
\end{tabular} \end{center}\begin{center} \begin{tabular}{|p{1.5cm}||p{2.5cm}||p{2.5cm}|p{2.5cm}||p{2.5cm}|p{2.5cm}|} \hline\multicolumn{2}{|c||}{\textbf{ Paris Centre }} & \multicolumn{2}{|c||}{Individual Target Related Performance} & \multicolumn{2}{|c|}{Industry Target Related Performance} \\
\hline 
Key Indicator & Number of units (systems for SSPI) with qualified results & Number of units (systems for SSPI) achieving individual target & Percentage of regional centre units achieving individual target & Number of units achieving industry target & Percentage of regional centre units achieving industry target \\
\hline 
FLR & 126 & 104 & 82.5 & 63 & 50 \\
\hline 
CRE & 126 & 106 & 84.1 & 88 & 69.8 \\
\hline 
TISA & 70 & 43 & 61.4 & 23 & 32.9 \\
\hline 
US7 & 126 & 103 & 81.7 & 65 & 51.6 \\
\hline 
SSPI & 310 & 310 & 100 & 126 & 100 \\
\hline 
\end{tabular} \end{center}\begin{center} \begin{tabular}{|p{1.5cm}||p{2.5cm}||p{2.5cm}|p{2.5cm}||p{2.5cm}|p{2.5cm}|} \hline\multicolumn{2}{|c||}{\textbf{ Tokyo Centre }} & \multicolumn{2}{|c||}{Individual Target Related Performance} & \multicolumn{2}{|c|}{Industry Target Related Performance} \\
\hline 
Key Indicator & Number of units (systems for SSPI) with qualified results & Number of units (systems for SSPI) achieving individual target & Percentage of regional centre units achieving individual target & Number of units achieving industry target & Percentage of regional centre units achieving industry target \\
\hline 
FLR & 61 & 53 & 86.9 & 34 & 55.7 \\
\hline 
CRE & 61 & 60 & 98.4 & 50 & 82 \\
\hline 
TISA & 34 & 34 & 100 & 31 & 91.2 \\
\hline 
US7 & 61 & 44 & 72.1 & 36 & 59 \\
\hline 
SSPI & 156 & 154 & 98.7 & 59 & 96.7 \\
\hline 
\end{tabular} \end{center}
%------------------------------------------------------------
\clearpage
\section*{Long-Term Targets Performance Bar Charts}
\subsection*{Number of Units Not Meeting Individual Target per Regional Centre}
%------------------------------------------------------------
\includegraphics{Fig-IdvCharts}

\subsection*{Number of Units Not Meeting Industry Target per Regional Centre}

\includegraphics{Fig-IdsCharts}


\section*{Long-Term Target Performance Trends \footnote{Please see more detailed definition regarding new and updated targets and indicators on the last pages}}
\subsection*{Worldwide Trends (percentage of units achieving targets)}

\includegraphics{Fig-PerfTrendWW}

\subsection*{AC Trends (percentage of units achieving targets)}
\includegraphics{Fig-PerfTrendAC}

\subsection*{MC Trends (percentage of units achieving targets)}
\includegraphics{Fig-PerfTrendMC}

\subsection*{PC Trends (percentage of units achieving targets)}
\includegraphics{Fig-PerfTrendPC}

\subsection*{TC Trends (percentage of units achieving targets)}
\includegraphics{Fig-PerfTrendTC}

%------------------------------------------------------------
\section*{Long-Term Targets Performance Trends}
%------------------------------------------------------------


%\begin{center}
\begin{figure}[h]
\centering
\includegraphics[scale=0.4]{FLR_RC}
\end{figure}
\begin{figure}[p]
\centering
\includegraphics[scale=0.4]{CRE_RC}
\includegraphics[scale=0.4]{TISA_RC}
\end{figure}
\begin{figure}[p]
\centering
\includegraphics[scale=0.4]{US7_RC}
\includegraphics[scale=0.4]{SSPI_RC}
\end{figure}
%\end{center}

\subsection*{The Brief Performance Analysis}

\subsubsection*{Introduction}
The aim of this report is to define the most important issues for
nuclear industry and to suggest the potential ways for performance
improvement. This analysis is based on Performance Indicator data
since 2007.

\subsubsection*{The main issue for nuclear industry} 
The recent PI results show that \textbf{FLR} performance is the
biggest issue. The industry performance looks stable but
significantly below the industry objective for FLR (see the FLR
definition on the page \pageref{FLR}). The industry value for this key
indicator is 58\% however the objective is 75\%. Looking at the 
\textbf{US7} industry performance indicator, which is 61\% (with the same 75\%
objective),scrams appear to be the main influence to poor FLR performance.

The ratio of \emph{automatic scrams} (AS) for the last year is 67\%. The
PWRs have 81\% of the AS, PHWR - 47\%.

The worst averaged number of scrams per reactor for 2016 have PHWRs (1.1 scrams
per reactor(SPR)), the best one - PWR (0.39 SPR).

The main reason for automatic scrams is Signal (from command system),
for manual one - circuit failure (based on OE Reports).

Looking at Regional Centers (RCs) performance it seems that Paris and Tokyo
Centers have the lowest FLR performance (there are 63 and 27 units
respectively that didn't reach the industry target for FLR and 61 and
25 units that didn't meet the US7 target).

The best but insufficient performance for FLR was Moscow Center.

\paragraph{It is recommended} all the RCs focus on the scrams rate to
define and solve the main related issues. However they shouldn't
forget that inappropriate maintenance and operation culture make a
significant contribution to poor FLR performance. 

The ratio of Forced Energy Losses for the last year is 45\% out of
whole the Unplanned Energy Losses (UEL) and rate
of Outage Extension Energy Losses is 55\%. It means that more than a
half of the UEL is related to inappropriate outages and maintenance.

The ratio of UEL out of all the Energy Losses is 28\%.

We would also recommend that WANO create an Industry Working Group (PI IWG)
(separated section) to solve these issues.


\subsubsection*{The second issue} 
Continuous PI analysis showed that the second issue for the nuclear
industry is TISA performance. The whole WANO performance is about 71\%
that is slightly below the 75\% objective.

All the RCs have good performance however Paris Center
demonstrated the worst one (33\%) that is significantly lower than the
objective. There are 47 units that didn't meet
industry target for TISA in WANO PC.

\paragraph{The recommendation} might be that Paris Center focus on
the industrial safety issues. \emph{However} we are aware that
some RCs have a problem in getting correct ISA/CISA data from
their members. We would also suggest creating a PI IWG section for
solving the issue.

\subsubsection*{RCs performance} 

\emph{WANO AC} demonstrates good trends
for all the key indicators, however there is still some room for
improvement for FLR (individual and industry), CRE (individual), TISA
(individual), US7 (both) and SSPI (both).

\emph{WANO MC}: unfortunately we see a significant regression
in FLR (both objectives, not only related to the scrams rate) and SSPI
since 2014;  there is still some room for
improvement for CRE (individual), TISA (both), US7 (individual) and
SSPI (both).

Also we might underline two issues regarding data quality: (a) WANO
MC units do not report us any fuel related events if the fuel
manufacturer is a Russian company; however we can see a lot of fuel
related issues based on our FRI, and (b) most of WANO MC units
continue to report average (non-unit based) CRE.

\emph{WANO PC}: there is a little regression in FLR (individual)
and some room for improvement for FLR (industry), CRE (both), TISA
(both) and US7 (both).

We would like to underline that, despite the highest SSPI performance,
there are some questions in SSPI source data quality due to the number of
Operational Experience (OE) Reports that did not reflect the PI
Results.

\emph{WANO TC}: there is some room for improvement for FLR
(both), CRE (individual), US7 (both) and SSPI (both).


Regarding \emph{general issues,} we would suggest that RCs be more focused
on data quality, especially for generation losses and safety system
performance. The TISA data quality should be on focus as well. We would
suggest that RCs and LO provide annual plant visits to check source
data collection and to have a look at the resulting usage issues.

To provide a \emph{forecast analysis} regarding further trends and potential
issues for whole the nuclear industry and for units and plants we
would also suggest that WANO starts to develop in-depth PI and OE
analytical system based on the recent research in the artificial
intelligence (AI) area. It might be included into Technical
Specification for the updated PI Database and application.

More detailed information regarding charts and numbers were presented
here is available in the PI Report system or directly from WANO LO PI
team.


\appendix
\subsection*{Targets definition}

For most performance indicators, the industry values were developed based on 2007 industry results. The industry-level targets are based on 75 percent of the industry achieving the 2007 industry median values. This would mean that overall industry performance has improved, with an additional one-fourth of the industry units or stations achieving performance indicators results better than the 2007 industry median. The individual unit or station performance targets are based on all units and stations achieving results that are better than the 2007 lowest quartile values.
The safety system performance targets are based on a continuing reduction of the industry average safety system unavailability to below 2007 industry average values. The unit/station safety system targets are based on first keeping the unavailability below a threshold value (0.020 or 0.025 depending on the system) and also either maintaining or decreasing the individual unit/station safety system unavailability.

\subsubsection*{WANO Performance Indicator Targets 2020}

The 2020 long-term targets for the WANO Performance Indicators will now begin to be implemented, following the comparisons of the 2015 year-end performance against the 2015 targets.

The targets have been updated to reflect the fact that WANO members and the industry as a whole have met the challenging goals previously set by the organisation. Therefore, new goals for some indicators have had to be established. These were discussed within the PI programme teams across WANO, and then presented to and approved by the Executive Leadership Team (ELT) in its meeting in December 2014.

The most of the targets LTT-2020 are the same as those set out in LTT-2015, except for the following changes:
\begin{itemize}
\item{Collective Radiation Exposure targets for AGRs have been changed due to the change in plant conditions since 2000;}
\item{Personnel safety performance will be monitored against targets for a new Total Industry Safety Accident (TISA) indicator, which replaces the ISA indicator used for the 2015 targets;}
\item{The safety system performance indicator industry target is now based on the percentage of units achieving all the individual SSPI targets (100 percent);}
\item{Individual and industry targets have been added for the total unplanned scram rate per 7,000 hours critical indicator (US7). The US7 2020 industry target is based on the third quartile of the worldwide industry by reactor type.}
\end{itemize}

The LTT-2020 below should be met by the end of 2020, i.e., the unit and industry\footnote{Given the SSPI industry
  target definition for SSPI, the industry SSPI graph shows the percentage
  of units that have met \emph{all} the individual targets for the different
  safety systems (SP1, SP2 and SP5). For this percentage, the industry
  objective is 100\%} indicator values should be less than or equal to the values defined below for each indicator.

\begin{center}
\begin{tabular}{|p{5cm}|p{3cm}|p{3cm}|p{3cm}|}
\hline
\textbf{Indicator} & \textbf{Unit} & \textbf{Individual target} & \textbf{Industry target} \\
\hline
Operating Period Forced Loss Rate (FLR) & percent & 5.0 & 2.0 \\
\hline
Collective Radiation Exposure (CRE) & man-rem/man-Sievert & AGR: 10/0.10 & AGR: 5.0/0.05 \\
&&BWR: 180/1.80 & BWR: 125/1.25 \\
&&LWCGR: 320/3.20 & LWCGR: 240/2.40 \\
&&PHWR: 200/2.00 & PHWR: 115/1.15\\
&&PWR: 90/0.90 & PWR: 70/0.70 \\
\hline
Total Industry Safety Accident rate (TISA) & number per 200,000 hours worked & 0.50 & 0.20 \\
\hline
Safety System Performance Indicator (SSPI) & unavailability & SP1 and
                                                              SP2:
                                                              0.020
& 100 percent of worldwide units achieve the individual targets\\
&&SP5 (EAC): 0.025&\\
\hline
Unplanned total Scrams per 7,000 hours critical (US7) & rate & BWR, PWR, LWCGR: 1.0 & BWR, PWR, LWCGR: 0.5 \\
&&PHWR: 1.5 & PHWR: 1.0 \\
&&AGR: 2.0 & AGR: 1.0 \\
\hline
\end{tabular}
\end{center}

Unfortunately, system modifications need to be carried out before we can produce reports on the new long term targets in the usual manner. Therefore, WANO London Office provides all LTT-related reports and calculations until these modifications have been completed.

\subsubsection*{Forced Loss Rate (FLR)}\label{FLR}

The forced loss rate (FLR) is defined as the ratio of all unplanned forced energy losses during a given period of time to the reference energy generation minus energy generation losses corresponding to planned outages and any unplanned outage extensions of planned outages, during the same period, expressed as a percentage.
Unplanned energy losses are either unplanned forced energy losses (unplanned energy generation losses not resulting from an outage extension) or unplanned outage extension of planned outage energy losses.
Planned energy losses are those corresponding to outages or power reductions which were planned and scheduled at least 4 weeks in advance.

\subsubsection*{Collective Radiation Exposure (CRE)}

Collective radiation exposure, for purposes of this indicator, is the total external and internal whole body exposure determined by primary dosimeter (thermoluminescent dosimeter (TLD) or film badge), and internal exposure calculations. All measured exposure should be reported for station personnel, contractors, and those personnel visiting the site or station on official utility business.
Visitors, for purposes of this indicator, include only those monitored visitors who are visiting the site or station on official utility business.

\subsubsection*{Total Industrial Safety Accident Rate (TISA)}

This indicator is defined as the number of accidents for all plant personnel, including all staff, contractors, supplemental personnel, and all other non-utility personnel working onsite that result in one or more days away from work (excluding the day of the accident) or fatalities per 200,000 (TISA2) or per 1,000,000 (TISA1) man-hours worked. The selection of 200,000 man-hours worked or 1,000,000 man-hours worked for the indicator will be made by the country collecting the data, and international data will be displayed using both scales.

\subsubsection*{Safety System Performance (SSPI)}

The purpose of the safety system performance indicator is to monitor the readiness of important safety systems to perform certain functions in response to off-normal events or accidents. This indicator also indirectly monitors the effectiveness of operation and maintenance practices in managing the unavailability of safety system components.
The safety system performance indicator provides a simple indication of safety system unavailability that shows good correlation with results of system unavailability calculations using more precise system modelling techniques (e.g. fault trees). A low value of the safety system performance indicator indicates a greater margin of safety for preventing reactor core damage and less chance of extended plant shutdown due to failure of a safety system to function during an operational event.
However, the objective should not be to attain a safety system performance indicator (unavailability) value that is near zero over a long term. Rather, the objective should be to attain a value that, while low, allows for maintenance activities to help maintain system reliability and availability consistent with safety analyses.
The safety system performance indicator is defined for the many different types of nuclear reactors within the WANO membership. To facilitate better understanding of the indicator and applicable system scope for these different type reactors a separate section has been developed for each reactor type.

\subsubsection*{Unplanned Total Scrams Per 7,000 Hours Critical (US7)}

The indicator is defined as the sum of the number of unplanned automatic scrams (reactor protection system logic actuations) and unplanned manual scrams that occur per 7,000 hours of critical operation.
The value of 7,000 hours is representative of the critical hours of operation during a year for most plants. It provides an indicator value that typically approximates the actual number of scrams occurring during the year.

\end{document}
