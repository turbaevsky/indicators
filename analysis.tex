\subsection*{The Brief Performance Analysis}

\subsubsection*{Introduction}
The aim of this report is to define the most important issues for
nuclear industry and to suggest the potential ways for performance
improvement. This analysis is based on Performance Indicator data
since 2007.

\subsubsection*{The main issue for nuclear industry} 
The recent PI results show that FLR performance is the
biggest issue. The industry performance looks stable but
significantly below the industry objective for FLR (see the FLR
definition on the page \pageref{FLR}). The industry value for this key
indicator is 58\% however the objective is 75\%. Looking at the 
\textbf{US7} industry performance indicator, which is 61\% (with the same 75\%
objective),scramsappear to be the main influence to poor FLR performance.


Looking at Regional Centers (RCs) performance it seems that Paris and Tokyo
Centers have the lowest FLR performance (there are 63 and 27 units
respectively that didn't reach the industry target for FLR and 61 and
25 units that didn't meet the US7 target).

The best but insufficient performance for FLR was Moscow Center.

\paragraph{It is recommended} all the RCs focus on the scrams rate to define and solve the main related
issues. However they shouldn't forget that inappropriate maintenance
and operation culture make a significant contribution to poor FLR
performance. We would also recomment that WANO create an Industry Working Group
(PI IWG) (separated section) to solve these issues.


\subsubsection*{The second issue} 
Continuous PI analysis showed that the second issue for the nuclear
industry is TISA performance. The whole WANO performance is about 71\%
that is slightly below the 75\% objective.

All the RCs have good performance however Paris Center
demonstrated the worst one (33\%) that is significantly lower than the
objective. There are 47 units that didn't meet
industry target for TISA in WANO PC.

\paragraph{The recommendation} might be that Paris Center focus on
the industrial safety issues. \emph{However} we are aware that
some RCs have a problem in getting correct ISA/CISA data from
their members. We would also suggest creating a PI IWG section for
solving the issue.

\subsubsection*{RCs performance} 

\emph{WANO AC} demonstrates good trends
for all the key indicators, however there is still some room for
improvement for FLR (individual and industry), CRE (individual), TISA
(individual), US7 (both) and SSPI (both).

\emph{WANO MC}: unfortunately we see a significant regression
in FLR (both objectives, not only related to the scrams rate) and SSPI
since 2014;  there is still some room for
improvement for CRE (individual), TISA (both), US7 (individual) and
SSPI (both).

Also we might underline two issues regarding data quality: (a) WANO
MC units do not report us any fuel related events if the fuel
manufacturer is a Russian company; however we can see a lot of fuel
related issues based on our FRI, and (b) most of WANO MC units
continue to report average (non-unit based) CRE.

\emph{WANO PC}: there is a little regression in FLR (individual)
and some room for improvement for FLR (industry), CRE (both), TISA
(both) and US7 (both).

We would like to underline that, despite the highest SSPI performance,
there are some questions in SSPI source data quality due to the number of
Operational Experience (OE) Reports that did not reflect the PI
Results.

\emph{WANO TC}: there is some room for improvement for FLR
(both), CRE (individual), US7 (both) and SSPI (both).

Regarding \emph{general issues,} we would suggest that RCs be more focused
on data quality, especially for generation losses and safety system
performance. The TISA data quality should be on focus as well. We would
suggest that RCs and LO provide annual plant visits to check source
data collection and to have a look at the resulting usage issues.